\documentclass[12pt,a4paper]{report}

\usepackage{epcc}
\usepackage{graphics}

% This example file shows how a thesis can be laid out using Latex. It
% does not use any special local features so should be portable to other
% places.
% 
% When producing draft copies of a thesis you may want to print only
% selected pages of the thesis. To do this use the command
% 
% dvips -f -p 4 -n 3 myfile.dvi | lpr
% 
% where -p 4 means start printing at page 4 (ie the page that will be
% numbered 4, not necessarily the 4th page) and -n 3 means print 3 pages.
% This example will print pages 4, 5 and 6.
% 
% If you want to print the thesis and also save paper you can print more
% than one page on each sheet of paper. Use the command
% 
% dvips -f myfile.dvi | psnup -2 | lpr
% 
% to print 2 pages per sheet. psnup can take values 2, 4, 8, or 9.
%
% To produce a PDF version you can create a PostScript copy first
%
% dvips -f myfile.dvi > myfile.pdf
%
% and then convert it
%
% distill myfile.ps
%
% or you can go straight to PDF
%
% pdflatex myfile
%
% Note that pdflatex expects all included figures to be in PDF too. See
% the includegraphics command below.


% This document contains many cross-references and forward references,
% eg in constructing a table of contents, so Latex may need to be run
% twice to get all the references correct. If you need to run Latex twice
% you may get the warning:
% 
% LaTeX Warning: Label(s) may have changed. Rerun to get cross-refSerences right.


% the following 4 lines are the content of the smallmargins.sty file
% but including them explicitly makes this more portable.
%AC%\oddsidemargin=0.1in
%AC%\topmargin=-0.5in
%AC%\textheight=9in
%AC%\textwidth=6.25in

%AC%\parskip 10pt
%AC%\parindent 0in

\begin{document}

%AC%\pagestyle{myheadings}
%AC%\markright{D.~S.~Henty}

%\title{A Latex thesis example}
%\author{D.~S.~Henty}
%\date{\today}

%\maketitle

\pagenumbering{roman}

\title{Hawk-i HPC CLoud Benchmark Tool}
\author{Visakh Muraleedharan}
\date{\today}

\makeEPCCtitle

\thispagestyle{empty}

\vspace{12cm}

\begin{center}

\large{MSc in High Performance Computing}

\large{The University of Edinburgh}

\large{Year of Presentation: 2012}

\end{center}

\newpage

\begin{abstract}
Scientific computing unravel the mysteries of science by constructing mathematical models and numerical algorithms. This required massive computational power and High-Performance Computing (HPC) solutions like cluster and grid have been the answer for massive computing to these needs for a long time. These HPC solutions are hard and expensive to setup, maintain and use. \emph{Cloud computing} is a model of delivering the existing compute infrastructure where computation and storage can be dynamically provisioned on a pay as you go model. Using a vendor cloud service like Amazon Web Service (AWS) can significantly reduce the effort to access these on-demand high performance resources.

In this work we create a platform to study how useful Amazon EC2 cloud computing can be for scientific applications. 
We classify the applications based on the concept of computational \emph{motif}\cite{asanovic2008parallel}  
\end{abstract}

\pagenumbering{roman}

\tableofcontents
\listoftables
\listoffigures

\begin{titlepage}
\vspace*{2in}
% an acknowledgements section is completely optional but if you decide
% not to include it you should still include an empty {titlepage}
% environment as this initialises things like section and page numbering.
\section*{Acknowledgements}

\end{titlepage}

\pagenumbering{arabic}



\section{Introduction}  
In this dissertation, we are trying to prove that a live micro benchmark tool can be useful to cloud users to estimate 
the time and cost of execution before putting significant initial efforts and skills to port the applications to the cloud infrastructure. 
We also study the possibilities of classifying micro benchmarks based on 13 emph{motifs}\cite{} or patterns.  
Our experiment uses two of these patterns, spectral methods and N-body simulation, to test the concept of live benchmark tool.   

\section{Cloud computing}  
In 1960 John McCarthy predicted that "computation may someday be organised as a public utility" \cite {dikaiakos2009cloud}. 
Availability of high capacity networks and low cost hardware as well as the adoption of hardware virtualisation technology 
made this prediction come true. These technologies and delivery models called "cloud computing" enabled a large community of 
end-users to obtain computing and storage capacity on-demand.  
There are several aspects in cloud computing which makes it attractive to users. The most importan one is the illusion of 
infinite computing resources on demand which reduces the user's effort to plan ahead for resource provisioning. 
Another feature is that cloud computing allow users to start small in a test environment and grow bigger without any upfront 
commitments to service providers. Finally, from a pricing point of view, provision to pay only for the computing, storage or network 
resources that you use makes cloud a cost effective solution for users \cite {fox2009above}.  

\subsection{HPC in the cloud}  
Cloud computing started in context of web applications and enterprise systems that have completely different requirements 
when compared to high performance computing applications like scientific computing. Traditionally these applications rely 
on HPC centers with dedicated systems connected through high bandwidth interconnects and uses parallel file systems.   
Cloud provides the ability to manage or modify the software environment and get on-demand access to virtual resources 
which makes it attractive to HPC users. The cloud resources can be used to replace the existing systems or supplement them. 
One good example is the cloud storage provided by
    
http://science.energy.gov/~/media/ascr/pdf/program-documents/docs/Magellan_Final_Report.pdf  

Introduction Paragraph, comparison with Clusters  
\subsection{Advantages}  
\subsection{Limitation}  
Network Limitations, Availability, Stability of performance  
Benifits  

\subsection{Cloud Computing Deployment models}  
A cloud being a data center hardware and software can be deployed in several ways depending on several aspects like cost effectiveness, privacy and security. The four deployment models of cloud computing are:  

\underline{ Public Cloud}  
When the cloud is available to the general public in a pay as you go manner, it is called a public cloud. 
Public cloud infrastructure is owned and maintained by vendors that can support very large infrastructure and hence they are also known as vendor clouds. 
Amazon Web Services\cite{} which is studied in this project would fall in this category. 
Some of the other major companies offering cloud services are Rackspace\cite{}, Microsoft\cite{} and Google\cite{}. 
Typically, end-users registered for this service using a credit card and they are provided access to this cloud infrastructure via internet.  

\underline{Private Cloud}  
A cloud infrastructure operating for a single organisation is a private cloud. This can be managed by the organisation or by a third party. 
Typically a private cloud is hosted within the organisation network but they can also be hosted externally. 
To provide virtualisation to users cloud software stacks like OpenNebula, Eucalyptus and Openstack are used. 
Private cloud provides most of the benefits of public clouds while avoiding the issues concerning security of data and performance of public clouds. 
Even though the organisation have to setup and manage the infrastructure,  it can be less expensive than public clouds\cite{Magellan_Final_Report.pdf}.

\underline{ Community Cloud}  
When a private cloud infrastructure is deployed by two or more organisations having similar requirements, it becomes a community cloud. 
The cost of operating the cloud is shared by the organisations in the community. 
Cost need to directly translate to currency in a community cloud, it can be credits based or fixed usage based on the agreement between the organisations.  

\underline{Hybrid Cloud}  
To benefit from the advantages of multiple deployment models, the two or more cloud infrastructures can be bound together to form a hybrid cloud. It requires onsite resources accessible without internet that provides flexibility and remote server based infrastructure to provide scalability. Hybrid clouds can be very useful with the implementing concepts like cloud bursting where the software application running on an internal organisational cloud is dynamically ported to a public cloud to address the increase in resource demand. To enable portability of applications in hybrid model, the independent infrastructures should be technology compliant.  


\subsection{Cloud computing service models}  
Cloud computing   
\subsubsection{IaaS}  
\subsubsection{PaaS}  
\subsubsection{Saas}  


\section{Amazon Web Service}  
\subsection{Amazon Machine Instance}  
\subsection{Types of Instances}  
\subsection{Amazon EC2}  
different instances chart and limitations of microinstance.  
\subsubsection{EC2 cluster compute}  
\subsection{Amazon Cluster compute}  
\section{Parallel Applications and 13 Dwarfs}  
Explain the 13 dwarfs and under the two sections explain N-body and FFT  
\subsection{N-Body Methods}  
MD, About the dwarf and application types optimisation, n2 and nlogn  
Communication pattern!!!  
\subsection{Spectral Methods}  
In scientific computing, a class of techniques used to numerically solve differential equations involving the use of Fast Fouries Transform are called spectral methods  
Fourier transform  
Jean Baptiste Joseph Fourier (1768-1830) first employed what we now call Fourier transforms whilst working on the theory of heat   
Linear transform which takes temporal or spatial information and converts into information which lies in the frequency domain  
Who would use Fourier Transforms?   
• Physics   
– Cosmology (P3M N-body solvers)   
– Fluid mechanics   
– Quantum physics   
– Signal and image processing   
– Antenna studies   
– Optics   
• Numerical analysis   
– Linear systems analysis   
– Boundary value problems   
– Large integer multiplication (Prime finding)   
• Statistics   
– Random process modelling   
– Probability theory  
{\underline Discrete Fourier Transform}  
The discrete Fourier transform of N complex points fk is defined as   
Communication pattern!!!  
 



\chapter{Live Benchmark Tool Setup}
\section{System Design}
\subsection{Architecture}
\subsection{Sequence diagram}
\subsection{Database Design}

\section{Instance types Used}
Refer to graph in Background
Why these types?
\section{Sun Grid Engine Cluster}
Why sungrid Engine
\section{SGE Clustering in Amazon cloud}
  benchmark to show how cluster instances are faster than normal high cpu instances
\section{Web interface}
Describe in detail
\subsection{Admin panel}
\subsection{User Dashboard}

\chapter{Results and Analysis}
\section{Serial}
for each instance type
time to result, execution time, increasing problem size
N-body and FFT

\section{Parallel program}
execution time, increasing problem size, number of cores
N-body and FFT
\section{Stability of results}
History of execution, snapshot from dashboard

\chapter{Conclusions}


\appendix
% the appendix command just changes heading styles for appendices.

\chapter{Cluster Computing Setup}
\subsection{Using Starcluster}
\subsection{Cluster Management package}
\subsection{Using Sun Grid Engine}
\subsection{Creating Dashboard}
\chapter{Benchmarking  programs}
\subsection{Spectral Methods}
\subsection{N-body}

\bibliographystyle{plain}
\bibliography{dissertation}



\end{document}
